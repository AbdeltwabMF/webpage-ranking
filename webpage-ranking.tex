\documentclass{scrartcl}

\usepackage{geometry}
\geometry{
	paper=a4paper, % Paper size
	top=2.5cm, % Top margin
	bottom=2.5cm, % Bottom margin
	left=2.5cm, % Left margin
	right=2.4cm, % Right margin
	headheight=0.75cm, % Header height
	footskip=1.5cm, % Space from the bottom margin to the baseline of the footer
	headsep=0.75cm, % Space from the top margin to the baseline of the header
	%showframe, % Uncomment to show how the type block is set on the page
}

\usepackage{tabularx}
\usepackage{booktabs}
\usepackage{blindtext}
% Character encoding
\usepackage[T1]{fontenc}
\usepackage[utf8]{inputenc}
% Mathematics packages from AMS
\usepackage{amsmath, amsfonts, amsthm, amssymb}
\usepackage{braket, nicefrac}
% International System of Units
\usepackage{siunitx}
% Lists / numbers
\usepackage{enumitem, multicol}
% Figure insertions
% Use option [H] to force the placement of a figure
\usepackage{graphicx, float}
\usepackage{keystroke}
\usepackage{pgfplots}\usepgfplotslibrary{units}\pgfplotsset{compat=1.16}
% Depth of the ToC
\setcounter{tocdepth}{3}
% Hyperlink References
\usepackage{hyperref}
% Storage Path for images
\graphicspath{{images/}}

\setkomafont{disposition}{\normalfont\bfseries}

% Environments
\renewenvironment{abstract}{
    \begin{center}
    {\Large \textbf{Abstract}}
    \vspace{0.5cm}
    \par\itshape
    \begin{minipage}{0.8\linewidth}}{\end{minipage}
    \noindent\ignorespaces
    \end{center}
}

\newenvironment{preface}{
  \begin{center}
    {\Large \textbf{Preface}}
    \vspace{0.5cm}
    \par
    \begin{minipage}{0.8\linewidth}}{\end{minipage}
    \noindent\ignorespaces
  \end{center}
}


\begin{document}
%Title of the report, name of coworkers and dates (of experiment and of report).
\begin{titlepage}
	\centering
	\includegraphics[width=0.2\textwidth]{al-azhar.png}\par\vspace{12pt}
	{\LARGE Al-Azhar University \par}\vspace{3pt}
	{\Large Faculty of Engineering \par}
	{\Large Computers \& Systems Engineering Department \par}\vspace{12pt}
	\vfill
	{\huge\bfseries\scshape Web Page Ranking \par}\vspace{8pt}
  {\scshape Seo Suggestion Search Engine Project \par}\vspace{8pt}
  {\itshape (Software Design Report) \par}
	\vfill
	{\Large\texttt{Contributors:} \\[12pt]
    \Large\itshape\ttfamily
    \begin{tabular}{ll}
      55 & Abd El-Twab M. Fakhry \\
      56 & Abd El-Hameed Hassan \\
      57 & Abd El-Khalek Alashker \\
      58 & Abdurrahman Khalefa \\
      59 & Abdurrahman Gamal \\
      60 & Abdurrahman Ramadan \\
    \end{tabular}
  }\par
	\vspace{1cm}
	\vfill
  {\Large\texttt{Supervised by:} \\
	\texttt{Dr. Muhammad Atef} \par}
  \vfill
  {\large \today \par}
\end{titlepage}

\newpage

\begin{preface}
  The purpose of a preface is to persuade your readers that they should read the rest of your written work. Keep it short. Describe your background and credentials. Discuss what inspired the project described in this report. Explain who your target audience is, and tell the reader why this project is important to them.
\end{preface}\vspace{1cm}

\begin{abstract}
  A strong abstract sums up your work in very few sentences:
  (i) state the problem you are addressing;
  (ii) say why it’s an interesting problem, and which issues are hard to tackle;
  (iii) give your approach towards solving the problem;
  (iv) say why and how well your approach solves the problem.
\end{abstract}\vspace{1cm}

\newpage

\tableofcontents

\newpage

\section{Introduction}

Your introduction briefly explains the problem you address, and what you've achieved towards solving the problem. It's an edited and updated version of your context and objectives from your topic outline document.

\section{User Stories}

\subsection{User Login}

\begin{table}[H]
  \caption{User Story 1}
  \begin{tabular}{p{0.45\linewidth} | p{0.45\linewidth}}
    \toprule
    User Story Code & US1 \\
    \midrule
    Name & User login \\
    \hline
    Statement & As a user, I want to login into the system, So that I can search for specific topics. \\
    \hline
    Priority & HIGH \\
    \hline
    Direct actors & User \\
    \hline
    Pre-Conditions & {
                     \begin{enumerate}
                     \item The user has internet access.
                     \item The user has a browser.
                     \item The user opens the login page in the browser.
                     \end{enumerate}
                     } \\
    \hline
    Post-Conditions & Application functionalities will be available for a user to use. \\
    \hline
    % Main Success Section = Use-Cases >> Diagrams
    Senario & {
              \begin{enumerate}
              \item The user fills in username and password fields.
              \item The user clicks on the login button.
              \item The system will validate the username and password. If correct, log in; otherwise, stay on the current login page.
              \item The user can search for a specific term or enter a live URL for keyword suggestions.
              \item The system logs both correct and wrong login.
              \end{enumerate}
              } \\
    \bottomrule
  \end{tabular}
\end{table}

\subsection{Administrator Login}

\begin{table}[H]
  \caption{User Story 2}
  \begin{tabular}{p{0.45\linewidth} | p{0.45\linewidth}}
    \toprule
    User Story Code & US2 \\
    \midrule
    Name & Administrator login \\
    \hline
    Statement & As an administrator, I want to login into the system, So that I can enter the admin area. \\
    \hline
    Priority & HIGH \\
    \hline
    Direct actors & Administrator \\
    \hline
    Pre-Conditions & {
                     \begin{enumerate}
                     \item The administrator has internet access.
                     \item The administrator has a browser.
                     \item The administrator opens the Admin login page in the browser.
                     \end{enumerate}
                     } \\
    \hline
    Post-Conditions & Application functionalities will be available for an administrator to use. \\
    \hline
    % Main Success Section = Use-Cases >> Diagrams
    Senario & {
              \begin{enumerate}
              \item The administrator fills in username and password fields.
              \item The administrator clicks on the login button.
              \item The system will validate the username and password. If correct, log in; otherwise, stay on the current login page.
              \item The administrator can feed the database website URLs and their metadata.
              \item The system logs both correct and wrong login.
              \end{enumerate}
              } \\
    \bottomrule
  \end{tabular}
\end{table}

\subsection{Website URLs}

\begin{table}[H]
  \caption{User Story 3}
  \begin{tabular}{p{0.45\linewidth} | p{0.45\linewidth}}
    \toprule
    User Story Code & US3 \\
    \midrule
    Name & Website URLs \\
    \hline
    Statement & As an administrator, I want to feed the database website URLs and their metadata, So that the system can use them to search the user query. \\
    \hline
    Priority & HIGH \\
    \hline
    Direct actors & Administrator \\
    \hline
    Pre-Conditions & {
                     \begin{enumerate}
                     \item US2.
                     \end{enumerate}
                     } \\
    \hline
    Post-Conditions & - \\
    \hline
    % Main Success Section = Use-Cases >> Diagrams
    Senario & {
              \begin{enumerate}
              \item The administrator feeds the database a new row with the website URL and its metadata.
              \end{enumerate}
              } \\
    \bottomrule
  \end{tabular}
\end{table}

\subsection{Search Box}

\begin{table}[H]
  \caption{User Story 4}
  \begin{tabular}{p{0.45\linewidth} | p{0.45\linewidth}}
    \toprule
    User Story Code & US4 \\
    \midrule
    Name & Search Box \\
    \hline
    Statement & As a user, I want a search box, So that I can write down my search query. \\
    \hline
    Priority & HIGH \\
    \hline
    Direct actors & User \\
    \hline
    Pre-Conditions & {
                     \begin{enumerate}
                     \item US1.
                     \end{enumerate}
                     } \\
    \hline
    Post-Conditions & - \\
    \hline
    % Main Success Section = Use-Cases >> Diagrams
    Senario & {
              \begin{enumerate}
              \item The user write down his/her search query in the search box.
              \item The user clicks the search button.
              \item The system uses this query and matches it with the content provided in URLs fed in the database.
              \item The system generates a list of related URLs.
              \item The user can then click on any URL in the generated list, and the URL will open the corresponding website.
              \item The system increases the number of visits to that URL in the database.
              \item The system goes to the web page URL to fetch its metadata and gives points according to page errors.
              \item The system updates the page's rank in the database according to the points the page gains and the number of visits.
              \end{enumerate}
              } \\
    \bottomrule
  \end{tabular}
\end{table}

\subsection{SEO Suggestions}

\begin{table}[H]
  \caption{User Story 5}
  \begin{tabular}{p{0.45\linewidth} | p{0.45\linewidth}}
    \toprule
    User Story Code & US5 \\
    \midrule
    Name & SEO Suggestions \\
    \hline
    Statement & As a user, I want to search a live URL, So that I can get suitable keywords and meta descriptions. \\
    \hline
    Priority & REGULAR \\
    \hline
    Direct actors & User \\
    \hline
    Pre-Conditions & {
                     \begin{enumerate}
                     \item US1.
                     \end{enumerate}
                     } \\
    \hline
    Post-Conditions & - \\
    \hline
    % Main Success Section = Use-Cases >> Diagrams
    Senario & {
              \begin{enumerate}
              \item The user write down a live URL in the search box.
              \item The user clicks the search button.
              \item The system goes and scans the web page URL, extracts its meta-features.
              \item The system then lists out suitable keywords related to the contents of the web page URL and a meta-description.
              \end{enumerate}
              } \\
    \bottomrule
  \end{tabular}
\end{table}

\section{Use-cases}

\subsection{Actors}

\subsubsection{System Administrator}

\subsubsection{User}

\section{ERD Diagram}

\section{Testing Script}

\section{Conclusion}

\newpage

\bibliographystyle{IEEEtran}
\bibliography{references}


% \newpage

% To create Appendix with additional stuff
% Put data files, CAD drawings, additional sketches, etc.

% \appendix
% \section{Appendix}

\end{document}
